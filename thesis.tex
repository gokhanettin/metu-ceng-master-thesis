\documentclass[chaparabic,ceng,ms,12pt,oneandhalf]{metu}
\usepackage{appendix}
\usepackage{longtable}
\usepackage[pdftex,pagebackref=true]{hyperref}
\usepackage[all]{hypcap}
\usepackage{todonotes}
\usepackage{graphicx}
\graphicspath{ {./images/} }
\usepackage[figuresright]{rotating}
\usepackage{xy}
\usepackage{booktabs}
\usepackage{pifont}
\usepackage{color}
\usepackage{listings}
\usepackage{pdfpages}
\usepackage{array}
\usepackage{algorithm}
\usepackage{algpseudocode}
\usepackage{float}
\usepackage{caption}
\usepackage{lastpage}
\usepackage{afterpage}
\usepackage{lipsum}
\usepackage{adjustbox}
\usepackage{rotating}

% \usepackage{graphicx}
\usepackage{amsmath,amssymb} % define this before the line numbering.
\numberwithin{equation}{chapter}
% \usepackage{ruler}
\usepackage{color}
% \usepackage{cite}
% \usepackage[utf8x]{inputenc}
% \usepackage{footnote}
% \makesavenoteenv{tabular}
% \makesavenoteenv{table}

\renewcommand{\sectionautorefname}{\S}
\renewcommand{\subsectionautorefname}{\S}

\newcommand{\norm}[1]{\left\lVert#1\right\rVert}

\captionsetup{belowskip=12pt,aboveskip=8pt}
\newcommand{\tab}{\hspace*{2em}}
\DeclareGraphicsExtensions{.pdf,.png,.jpg}


\usepackage{amsmath}
\usepackage{siunitx}
\usepackage{textcomp}
\usepackage{subcaption}


\usepackage{tikz}
\usepackage{mathtools}
% \usepackage{rotating}
%\PassOptionsToPackage{figuresright}{rotating}

\DeclarePairedDelimiter\ceil{\lceil}{\rceil}
\DeclarePairedDelimiter\floor{\lfloor}{\rfloor}


\newcommand{\EA}[1]{\textcolor{red}{[EA: #1]}}

% Name and Surname
\author{Gökhan Karabulut}
% Thesis Title English and Turkish
\title{Mini Autonomous Car Architecture for Urban Driving Scenarios}
\turkishtitle{Şehir İçi Sürüş Senaryoları için Mini Otonom Araç Mimarisi}

\date{September 2019}

% prof : Prof. Dr.
% assocprof : Assoc. Prof. Dr.
% assistprof : Assist. Prof. Dr.
% dr : Dr.
%
% Director of Institute
\director[prof]{Halil Kalıpçılar}
% Head of Department
\headofdept[prof]{Halit Oğuztüzün}
%
% Supervisor : English and Turkish
\supervisor[prof]{Tolga Can}
% \turkishsupervisor{  } %if you will hard-code the academic title
%
% Affiliation of Supervisor in English and possibly in Turkish
\departmentofsupervisor{Department of Computer Engineering, METU}

\cosupervisor[assistprof]{Selim Temizer}
\departmentofcosupervisor{Dept. of Computer Science, Nazarbayev Univ.}
%
% Committee Members
% In general members are sorted according to their academic titles
%
% Proffesors (1)
% Associate Professors (2)
% Assistant Professors (3)
% Other (4)
%
% IMPORTANT:  All affiliatons should fit in a single line
% If affiliation line is broken into two lines you should shorten the affiliation by using
% abbrevations or any other means
%
% First committee member should be the chair of examining committee
% Typically the chair is one of the highest ranked committee members
% Ask your supervisor if you are not sure
\committeememberi[assocprof]{Yusuf Sahillioğlu}
\affiliationi{Department of Computer Engineering, METU}
% Second committee member is always your supervisor
\committeememberii[prof]{Tolga Can}
\affiliationii{Department of Computer Engineering, METU}
% If you are an M.Sc. student and your Co-Supervisor is in your
% examination committee, then third committee member is always your co-supervisor
%
% IMPORTANT: If you are Ph.D. student your co-supervisor can not be in your
% examination committee.

% \def\@proftitlename{Prof. Dr.}\def\@tproftitlename{Prof. Dr.}
% \def\@assocproftitlename{Assoc. Prof. Dr.}\def\@tassocproftitlename{Doç. Dr.}
% \def\@assistproftitlename{Assist. Prof. Dr.}\def\@tassistproftitlename{Yrd. Doç. Dr.}
% \def\@drtitlename{Dr.}\def\@tdrtitlename{Dr.}

\committeememberiii[assistprof]{Mehmet Tan}
\affiliationiii{Department of Computer Engineering, TOBB-ETU}
% Fourth committee member
% \committeememberiv[assistprof]{Gülşah Tümüklü Özyer}
% \affiliationiv{Computer Engineering, Atatürk University}
% Fifth committee member
% \committeememberv[assistprof]{Jüri}
% \affiliationv{JüriBölüm, Ankara University}
%
% Keywords : English & Turkish, Comma seperated
\keywords{
  autonomous car,
  traffic scene parsing,
  traffic sign classification,
  optimal trajectory planning,
  path tracking
}
\anahtarklm{
  otonom araç,
  trafik sahnesi ayrıştırma,
  trafik işareti sınıflandırma,
  optimal yörünge planlama,
  yol takibi
}
%
% Abstract in English
%
\abstract{
  Autonomous cars capable of driving in city traffic has been long studied in
  decomposed architectures consisting of perception, planning, and control
  components. Recent advances in deep learning techniques considerably
  contributed to the perception component of this approach. These advances also
  led to evolution of other approaches such as end-to-end learning of steering
  commands and learning driving affordances from camera images. Though the
  latter approaches are promising to simplify the overall architecture, the
  former is found more persuasive constituting the majority of today's
  state-of-the-art, market-oriented driverless cars. However, studies on
  small-scale autonomous cars, which can be considered as low-cost and rapid
  prototyping platforms, are not on a par with research on the modern
  decomposed architectures. These studies often remain limited to end-to-end
  approaches or resort to traditional image processing techniques in
  over-simplified traffic scenarios. In this thesis, we present a decomposed
  architecture for small-scale cars covering extended traffic scenarios with
  seven traffic signs, traffic lights, lane changes, cloverleaf interchange,
  pedestrian crossings, and parking. To realize this architecture, we created
  segmentation and classification datasets. We trained two deep learning models
  for learning lane semantics and classifying traffic signs and lights. We
  developed a behavior planner to decide on the best behavior primitives for
  traffic scenes. Based on these behavior primitives, we implemented a
  trajectory planner to find optimal trajectories along the lanes and a
  controller to follow these trajectories. With our novel lane segmentation
  scheme, 97\% accurate classifier, robust planner and controller algorithms,
  we achieved successful drives on simulated and real courses.
}
%
% Turkish Abstract
%
\oz{
  Şehir trafiğinde sürüş yapabilen otonom araçlar algılama, planlama ve kontrol
  bileşenlerinden oluşan ayrışmış mimariler olarak uzun zamandır
  çalışılmaktadır. Derin öğrenme tekniklerindeki son gelişmeler bu yaklaşımın
  algılama bileşenine büyük ölçüde katkıda bulunmuştur. Bu gelişmeler ayrıca
  kamera görüntüsü üzerinden uçtan uca yönelim komutlarını ve sürüş
  sağlarlıklarını öğrenmek gibi yeni yaklaşımların gelişimini de beraberinde
  getirmiştir. Sonraki yaklaşımlar genel mimariyi sadeleştirmek için ümit
  verici olsa da bugünün en gelişmiş, pazara yönelik sürücüsüz araçlarının
  çoğunluğunu oluşturan ilk yaklaşım daha ikna edici bulunmaktadır. Bununla
  birlikte, düşük maliyetli ve hızlı prototip oluşturma platformları olarak
  düşünülebilecek küçük ölçekli otonom araçlar üzerinde yapılan çalışmalar,
  modern ayrışmış mimariler üzerine yapılan araştırmalarla aynı düzeyde
  değildir. Bu çalışmalar genellikle uçtan uca yaklaşımlarla sınırlı kalmakta
  veya aşırı basitleştirilmiş trafik senaryoları içinde geleneksel görüntü
  işleme yöntemlerine başvurmaktadır. Bu tezde, küçük ölçekli araçlar için
  yedi trafik işareti, trafik ışıkları, şerit değişikliği, yonca yaprağı
  kavşağı, yaya geçitleri ve park işlemi ile genişletilmiş trafik senaryolarını
  kapsayan ayrışmış bir mimari sunuyoruz. Bu mimariyi gerçekleştirmek için
  bölütleme ve sınıflandırma veri setleri oluşturduk. Şerit anlamlarını
  öğrenmek ve trafik levha ve ışıklarını sınıflandırmak için iki derin öğrenme
  modeli eğittik. Trafik sahnelerine göre en iyi davranış temellerine karar
  veren bir davranış planlayıcısı geliştirdik. Bu davranış temellerine
  dayanarak, şerit boyunca en uygun yörüngeleri bulan bir yörünge planlayıcısı
  ve bu yörüngeleri takip etmek için bir denetleyici gerçekledik. Özgün şerit
  bölümleme tasarımız, \%97 doğru sınıflandırıcımız, dayanıklı planlayıcı ve
  kontrolcü algoritmalarımızla benzetimli ve gerçek güzergahlarda başarılı
  sürüşler gerçekleştirdik.
}
%
% Dedication
\dedication{To those who cannot do without Vim key bindings.}
%
%
% Acknowledgements
\acknowledgments{
  I would like to start with the members of team Robocodes; Berkant Bayraktar,
  Berker Acır, Ilker Ayçiçek, Yunus Emre Saçma, and Asst. Prof. Dr. Selim
  Temizer. Their patience against all odds made this thesis possible. Lack of
  space and budget to build a mini race course did not stop them.  They had the
  courage to develop a mini autonomous car in a simulated environment and see
  it in a competition without ever testing in a real enviroment against
  competition rules. Thank you all.

  Supervisor of our team and co-supervisor of my thesis, Asst. Prof. Dr. Selim
  Temizer, did more than supervising. I truly enjoyed the enlightenment moments
  of how complicated-looking subjects turned into intuitive ideas after talking
  to him. I am grateful to him for encouraging me to study autonomous cars in
  which I had chance to expand my knowledge in machine learning and robotics,
  and gain hands-on experience in both fields. Aside from supervising the team
  and my thesis, I cannot thank him enough for his time and effort to find a
  sponsor to afford a mini autonomous car platform.

  I am thankful to our sponsor, Acasus, for providing us with a mini autonomous
  car platform while we were desperately looking for a sponsor.

  I would like to thank to my supervisor, Prof. Dr. Tolga Can, for his support
  and for making all the thesis procedures easy for me.

  I also would like to thank to my friends at Turkish Aerospace Industries,
  Inc. (TAI) for being tolerant to my unexpected absence when I had to take
  annual leave so as to work on my thesis.
}

%
% End of Personal and Introductory Information
%%%%%%%%%%%%%%%%%%%%%%%%%%%%%%%%%5
\begin{document}
% Preliminaries
\begin{preliminaries}
% If you are willing to use any custom stuff before Chapters, put it here
% Such as List of Abbreviations
% Check the abbreviations.tex for a template list of abbreviations

\begin{theglossary}{LONGESTABBRV}

\item[2D] 2 Dimensional
\item[3D] 3 Dimensional 

\end{theglossary}

% End of Preliminaries
\end{preliminaries}
%
% Latex content Goes Here
%
%

\setlength{\parindent}{0em}
\setlength{\parskip}{10pt}

% You can add as many chapters
\chapter{Introduction}
\label{chp:b1}

\section{Problem Definition}
 Lorem ipsum dolor sit amet, consectetur adipiscing elit. Proin sodales augue
 sit amet eros maximus, sed venenatis metus vehicula. Proin vel ligula porta
 justo iaculis ultrices non a ex. Sed lacinia bibendum feugiat. Integer mollis
 ac quam eu malesuada. Pellentesque malesuada facilisis metus quis condimentum.
 Nulla eget odio eget arcu elementum fringilla. Vestibulum eget pretium libero.
 Ut fermentum eleifend enim, nec facilisis sapien iaculis eu. Curabitur
 volutpat turpis risus, sed sollicitudin velit imperdiet sit amet.

Aliquam tristique dictum facilisis. Aliquam non erat sed ligula lobortis varius
id eu est. Suspendisse dapibus auctor diam ut ultrices. Lorem ipsum dolor sit
amet, consectetur adipiscing elit. Vestibulum leo lectus, ornare ac lectus
eget, placerat ultrices nulla. Praesent sit amet sodales nibh. Sed vestibulum
non lorem non scelerisque.

\section{Contributions}
Lorem ipsum dolor sit amet, consectetur adipiscing elit. Proin sodales augue
sit amet eros maximus, sed venenatis metus vehicula. Proin vel ligula porta
justo iaculis ultrices non a ex. Sed lacinia bibendum feugiat. Integer mollis
ac quam eu malesuada. Pellentesque malesuada facilisis metus quis condimentum.
Nulla eget odio eget arcu elementum fringilla. Vestibulum eget pretium libero.
Ut fermentum eleifend enim, nec facilisis sapien iaculis eu. Curabitur volutpat
turpis risus, sed sollicitudin velit imperdiet sit amet.

Aliquam tristique dictum facilisis. Aliquam non erat sed ligula lobortis varius
id eu est. Suspendisse dapibus auctor diam ut ultrices. Lorem ipsum dolor sit
amet, consectetur adipiscing elit. Vestibulum leo lectus, ornare ac lectus
eget, placerat ultrices nulla. Praesent sit amet sodales nibh. Sed vestibulum
non lorem non scelerisque.

\section{Organization}
%% 2D

\begin{itemize}
\item There is no one who loves pain itself, who seeks after it and wants to
    have it, simply because it is pain \item There is no one who loves pain
    itself, who seeks after it and wants to have it, simply because it is pain
\item There is no one who loves pain itself, who seeks after it and wants to
    have it, simply because it is pain
\end{itemize}

\chapter{Background and Related Work}
\label{chp:b2}

In this chapter, we review prominent autonomous driving architectures together
with the algorithms they are composed of. Along the way, we also briefly
discuss the historical development of self-driving cars, particularly two most
significant self-driving car competitions that paved the way for today's
driverless car technologies; DARPA Grand Challenge and DARPA Urban Challenge.

In DARPA Grand Challenge 2004, none of the 15 teams saw the finishing line of
the race course. In DARPA Grand Challenge 2005, Stanley, a robot car developed
by Stanford Racing Team, was the first car to complete the race course. Thrun
et al.\ \cite{Thrun2006StanleyTR} presents the details of the competition rules
and the software design of Stanley. According to Thrun et al., a description of
the race course was given to the participants in a DARPA-defined format, namely
RDDF, two hours before the race. The RDDF contained a list of longitudes,
latitudes, road segment widths and a list of speed limits associated with the
road segments. In addition, the autonomous cars didn't need to deal with
dynamic obstacles.

The authors state that Stanley's software is designed as a data processing
pipeline and processing nodes communicate through a publish/subscribe
mechanism. Stanley localizes itself on the RDDF by incorporating data from GPS,
GPS compass, IMU, and wheel encoders with UKF at 100 Hz. It performs terrain
analysis based on laser sensors and camera. For both data sources, the team
automatically creates datasets through human driving and apply machine learning
algorithms to classify the terrain into drivable and nondrivable regions. For
the laser readings, a set of parameters such as obstacle height threshold and
acceptance probability along with the Markov model parameters that capture the
process and measurement noise covariances are learned in a discriminative
fashion by coordinate ascent algorithm.

As opposed to laser terrain analysis, Stanley uses generative learning
algorithm for camera based terrain analysis. Drivable quadrilaterals (ahead of
the vehicle, extracted from laser data) are projected onto the camera image.
The pixels inside the quadrilateral are then used as training samples. From
these samples, Stanley learns and maintains a database of Gaussians in RGB
space that corresponds to wide variety of drivable surfaces. The range of a
laser sensor is shorter than that of the camera. On the other hand, vision
based terrain analysis is susceptible to color and lighting changes in the
environment. Therefore, Stanley uses the laser based terrain analysis for
steering control and vision based analysis for speed control so that the vision
module acts as an early warning system when an obstacle is ahead but not within
the range of the lasers.

Because the detailed race track is provided in RDDF, Stanley's main focus is
local obstacle avoidance rather than global planning. Though there are no lanes
in the course, Stanley introduces lateral offsets over the base trajectory
which is a smoothed path extracted from RDDF. Stanley basically plans a
trajectory to smoothly change into a lateral offset within the drivable path
for obstacle avoidance similar to the lane change in highway driving. The
planner finds a minimum cost trajectory by evaluating a cost function which is
subject to kinematic and dynamic constraints of the vehicle, distance to
obstacles, distance from the center of the road, and being on the course
corridor.

Stanley uses its own steering control algorithm to track the optimal trajectory
proposed by the path planner. Relying on the geometrical relation between the
car pose and the trajectory, Stanley minimizes the cross-track error, which
measures the lateral distance between the center of front axle and the nearest
point on the target trajectory. The geometrical relation is illustrated in
Figure \ref{figure:stanley-control}.

\begin{figure}[h]
  \centering
  \includegraphics[width=.8\textwidth]{figures/stanley-control.png}
  \caption[Stanley controller algorithm]{The geometrical relation between the
    trajectory and the car used by Stanley controller algorithm. (Figure
    source: Thrun et al.\ \cite{Thrun2006StanleyTR})}
  \label{figure:stanley-control}
\end{figure}

Nonlinear feedback function of cross-track error is given by the equation

\begin{equation}
  \delta(t) = \psi(t) + \arctan\frac{kx(t)}{u(t)},
  \label{eq:stanley-control}
\end{equation}

where $k$ is a gain parameter, $u(t)$ is the car speed, and $\psi(t)$ denotes
the orientation of the nearest trajectory segment relative to the car's
orientation. The intuition behind the controller is that as the cross-track
error $x(t)$ increases, the controller produces stronger steering angle
$\delta(t)$ towards the trajectory. Likewise, as the speed $u(t)$ increases,
the controller avoids sudden strong maneuvers. Hoffmann et al.\ studies the
Stanley controller algorithm in greater detail \cite{Hoffmann2007AutonomousAT}.

In 2007, DARPA Urban Challenge took place. Montemerlo et al.\
\cite{Montemerlo2009JuniorTS} present the competition details and the software
architecture of Junior, Stanford's another robot car and the second best car in
the challenge. This time rules were more complex including overtaking parking
or moving vehicles, precedence handling at intersections possibly with stop
signs, merging into fast moving traffic, left turns, parking and U-turns when
the road is completely blocked. Participants were provided with a road network
description file, or RNDF, which contained lane information, stop signs,
parking lots, and special checkpoints. In addition, the teams were also
provided with a high resolution aerial image of the race course so that they
can further improve the RNDF. In the competition, the vehicles were given
multiple missions as a sequence of checkpoints in the RNDF.

Like Stanley, Junior's software architecture is made of sensor interfaces,
perception, navigation, and drive-by-wire interfaces at the core. The design
is again based on data processing endpoints communicating with
publish/subscribe paradigm. Unlike Stanley, Junior's modules are far more
advanced. Junior's perception module segments the environment data into moving
vehicles and static obstacles. Its navigation features a global path planner
based on dynamic programming to find an optimum path to the mission checkpoints
from the current location of the car. Moreover, the navigation module handles
different driving scenarios with different planning algorithms.

It performs free-form navigation in parking lots, at U-turns or whenever the
car gets stuck for extended period of time. The free-form planner is
specifically developed for Junior and named hybrid A* by the Stanford Racing
Team. Hybrid A* associates discrete search space of regular A* with a
continuous state by performing forward simulations with different steering
angles and computes a score based on the continuous state. The continuous state
is represented by x-y position of the car, heading direction, and the
direction, either forward or reverse. Whereas the path found by the regular A*
and Field D* algorithms cannot be executed due to their discrete nature, hybrid
A* can find executable paths that accounts for the nonholonomic constaints of
the vehicle. Hybrid A* uses dual admissible heuristics. One heuristic is
nonholonomic without obstacles, and the other is holonomic with obstacles. Once
a solution is found, it is smoothed for a better driving experience. Extensive
study and experiments show that hybrid A* produces near-optimal solutions
\cite{Dolgov2010PathPF, Petereit2012Application}. Figure
\ref{figure:hybridastar-comparison} demonstrates the difference between regular
A*, Field D*, and hybrid A* algorithms.

\begin{figure}[h]
  \centering
  \includegraphics[width=.8\textwidth]{figures/hybridastar-comparison.png}
  \caption[A*, Field D* and Hybrid A* algorithms]{Left: Regular A* solutions
    pass through only the center of grids. Center: Field D* solutions can
    have arbitrary linear paths from cell to cell. Right: Hybrid A* associates
    a continuous state with each cell and computes a score for the continuous
  state. (Figure source: Dolgov et al.\ \cite{Dolgov2010PathPF})}
  \label{figure:hybridastar-comparison}
\end{figure}

Junior uses a different planner for normal on-road navigation. It performs
internal simulations with different steering parameters. The internal
simulations generate candidate trajectories with respect to a reference path.
This reference path is essentially the smoothed center of the lane obtained
from RNDF. The planner evaluates the candidate trajectories by a cost function
and finally selects the best trajectory. The cost function also regulates the
lane change or overtaking behavior of Junior. When the right lane is blocked,
the car chooses to shift left. When overtaking is complete, it steers back
to the right lane as it would be more costly to occupy the left lane.

Driving behavior of Junior is governed by a hierarchical finite state machine.
The state machine decides on the U-turns, handles intersection precedence and
stop signs, prevents the car from getting stuck, switches to parking navigation
in a parking lot or chooses the true planner for the current scenario in
general.

Werling et al.\ \cite{Werling2010OptimalTG} report that they generated optimal
trajectories in Frenet frame and tested on Junior without obstacles. They
also present their obstacle avoidance experiments in simulation. In their
method, they suggest integrating trajectory generation with a behavioral layer
that decides on the high level as to whether the car should keep a constant
velocity, follow the car in front with a constant distance, merge into
traffic or stop at a point. Figure \ref{figure:frenet-velocity-keeping}
demonstrates a velocity keeping instance in this approach.

\begin{figure}[h]
  \centering
  \includegraphics[width=.8\textwidth]{figures/frenet-velocity-keeping.png}
  \caption[A sample optimal trajectory generated in Frenet frame]{A sample
    optimal trajectory generated in Frenet frame in velocity keeping mode.
    Colors from red to yellow represent increasing lateral cost. Colors from
    grey to black represent increasing longitudinal cost. Green and light grey
    colors represent the optimal trajectory, which leads the car to the
    reference line and desired speed. (Figure source: Werling et al.\
    \cite{Werling2010OptimalTG})}
  \label{figure:frenet-velocity-keeping}
\end{figure}

Yoneda et al.\ \cite{Yoneda2018TrajectoryOA} further extend the method developed
by Werling et al.\ \cite{Werling2010OptimalTG} introducing an additional adjust
mode while switching from velocity keeping mode to distance keeping in an
affort to eliminate strong acceleration and deceleration during the mode
switching in quest of a more natural driving experience.

Fast forward to the present day, Autoware \cite{Kato2018AutowareOB}, being one
of the modern open source self-driving car platforms is based on ROS
\cite{Quigley2009ROSAO}. ROS is a commonly used, extensible, component based,
higly modular middleware framework with many reusable packages and
visualization tools that dramatically accelerated today's robot development and
prototyping processes. Unsurprisingly, publish/subscribe mechanism is at the
core of ROS communication patterns. Autoware implements perception,
decision-making, planning and path tracking capabilities. Figure
\ref{figure:autoware} illustrates the the Autoware architecture at a high
level.

\begin{figure}[h]
  \centering
  \includegraphics[width=.8\textwidth]{figures/autoware.png}
  \caption[Autoware high level architecture]{Autoware high level architecture
    and data flow. (Figure source: Kato et al.\ \cite{Kato2018AutowareOB})}
  \label{figure:autoware}
\end{figure}

Perception capabilities are made of localization, detection, and prediction
modules. For the localization, Autoware relies on high definition 3D maps. It
localizes itself by appling scan matching between the 3D map and LiDAR scans.
Therefore, a 3D map of the environment should be created beforehand using SLAM
techniques, in which scan matching is applied against previous LiDAR scans
instead of a 3D map such that a transformation between the LiDAR scans are
obtained and a cumulative point cloud is continually updated. Other perception
modules also rely on the localization. For example, the car is localized in
the 3D map, it projects 3D map features into the front view camera images to
define a ROI for traffic light detection and classification in order to
eliminate full image search on every image frame. The detection module supports
both deep learning and traditional image processing and machine learning
techniques. It features YOLO2 \cite{Redmon2016YOLO9000BF} and SSD
\cite{Liu2016SSDSS} models for detecting objects in the traffic such as traffic
signals, pedestrians, and other vehicles. In the prediction module, Autoware
associates detected objects with time, so that it estimates trajectories for
the moving objects which are then used in the planning modules. Based on the
perception modules, Autoware makes decisions in response to the environmental
changes. The decision-making scheme is captured in a finite state machine
similar to Junior.

Autoware features two sets of planners, a mission planner and motion planners.
The mission planner is responsible for creating a rough global path from the
current location to the destination in the map. Motion planners, on the other
hand, generate local trajectories taking the global plan as a reference. In
unstructured environments such as parking lots, hybrid A* is used similar to
Junior. For well-structured environment scenarios such as navigating on the
lanes, state lattice based algorithms are preferred. Pivtoraiko et al.\
\cite{Pivtoraiko2009DifferentiallyCM} introduce space lattice based planning.
The state lattice is made of motion primitives of a specific car. The motion
primitives are generated offline by a precise trajectory generator respecting
the mobility model of the vehicle such as steering limits and wheelbase. Then,
the lattice search space could be searched by D* algorithms for optimal
trajectories. This method was also successfully used in DARPA Urban Challenge
by the winner vehicle, Carnegie Mellon University's Tartan Racing for
navigating in unstructural environments \cite{Urmson2007TartanRA}. Figure
\ref{figure:state-lattice} illustrates a state lattice. McNaughton et al.\
\cite{McNaughton2011MotionPF} later extended this approach and applied it to
structural environments.

\begin{figure}[h]
  \centering
  \includegraphics[width=.8\textwidth]{figures/state-lattice.png}
  \caption[An example state lattice]{An example state lattice without reverse
    motions. (Figure source: Pivtoraiko et al.\
    \cite{Pivtoraiko2009DifferentiallyCM})}
  \label{figure:state-lattice}
\end{figure}

Autoware uses pure pursuit controller to generate low level steering commands
to execute the given trajectory from the motion planners. Kim et al.\
\cite{Kim2013SensorbasedMP} studies the controller with its geometrical
derivation and also give some useful pointers on tuning.

Backed by Baidu, Apollo is another open source autonomous driving platform with
its giant dataset \cite{Huang2018TheAD}. Similar to other decomposed
architectures, Apollo is also made of localization, perception, prediction,
routing, motion planner, and vehicle control components as shown in Figure
\ref{figure:apollo}. Like Autoware, Apollo also relies on high definition 3D
maps for location and perception \cite{Fan2018BaiduAE}.

\begin{figure}[h]
  \centering
  \includegraphics[width=.8\textwidth]{figures/apollo.png}
  \caption[Apollo high level architecture]{Apollo high level architecture and
    data flow. (Figure source: Fan et al.\ \cite{Fan2018BaiduAE})}
  \label{figure:apollo}
\end{figure}

The routing component finds a global plan from the current location
to a destination in the map like previous architectures; however, Apollo's
lane-based motion planner is quite different. Apollo does not directly use
this global plan as a reference path for trajectory generation, but rather it
generates multiple lane level reference lines from it taking traffic
regulations (e.g., traffic signs, signals and lane markings) and safety
measures into account. During lane level motion planning, Frenet frames are
constructed based on the given reference lines. Lane level path and speed
optimizers generate the optimal trajectories in Frenet frame for each lane.
Finally, a trajectory decider chooses the best trajectory for the maneuver
given the cost of each trajectory, car status, traffic regulations. This
approach allows for dealing with different traffic regulations that apply for
different lanes of the same road \cite{Fan2018BaiduAE}.

A different approach to autonomous cars is to learn a mapping from input images
to steering angle and speed commands in an end-to-end manner. Bojarski et al.\
\cite{Bojarski2016EndTE} were the first to apply this method to a real-sized
car with the modern deep learning techniques. They collected 72 hours data
with different cars in different weather and lighting conditions from various
places. For the data acqusition, they installed three cameras on the car behind
the windshield and recorded timestamped videos from the left, right and center
cameras along with the steering commands controlled by a human driver. During
training, they augmented the dataset by random shifting and rotating the images
and adjusting the recorded commands accordingly. The trained model then
successfully steered the car by using the images only from the central camera.
Figure \ref{figure:end-to-end-network} demonstrates the training and testing
steps of this approach.

\begin{figure}[h]
  \centering
  \begin{subfigure}[b]{1.0\linewidth}
    \includegraphics[width=\linewidth]{figures/end-to-end-training.png}
    \caption{}
  \end{subfigure}
  \begin{subfigure}[b]{1.0\linewidth}
    \includegraphics[width=\linewidth]{figures/end-to-end-inference.png}
    \caption{}
  \end{subfigure}
  \caption[End-to-end training and inference]{(a) End-to-end training scheme.
    (b) Steering command inference from raw camera images. (Figure source:
    Bojarski et al.\ \cite{Bojarski2016EndTE})}
  \label{figure:end-to-end-network}
\end{figure}

Bechtel et al.\ \cite{Bechtel2017DeepPicarAL} replicated the study
\cite{Bojarski2016EndTE} with a small-scale, low-cost platform using a web
camera and a Raspberry Pi 3 for inference. They conducted successful
experiments in a specially built test course for their RC car.

Do et al.\ \cite{Do2018RealTimeSC} also implemented a similar approach in
another RC car platform using Pi camera and Raspberry Pi 3 for inference.
Instead of learning steering angles in a regression model, they learned a
steering angle probability for discretized steering angle space. In addition to
basic lane following, they also learned to turn left or right when the
corresponding traffic sign is encountered. The traffic sign diameter was 15 cm
in their experiments.

There are several traffic scene segmentation datasets available. Currently,
the largest and the most comprehensive one is ApolloScape
\cite{Huang2018TheAD}. It is followed by Cityscapes \cite{Cordts2016TheCD},
KITTI \cite{Geiger2012AreWR}, and Mapillary Vistas \cite{Neuhold2017TheMV}
datasets. Because these datasets are created for real-sized cars on real roads,
they were not suitable for us, so we had to create our own segmentation
dataset. The existing traffic sign and signal classification datasets
\cite{Timofte2009MultiviewTS, Stallkamp2012ManVC, Shakhuro2016RussianTS,
Serna2018ClassificationOT, MaldonadoBascn2007RoadSignDA}, on the other hand,
was useful to train an initial classifier as they are mostly independent of the
scene and car size.

Sakai et al.\ \cite{Kim2013SensorbasedMP} present a collection of various
autonomous navigation algorithms implemented in Python Programming Language in
their basic forms. The collection includes aforementioned hybrid A*, Frenet
optimal trajectory planner, state lattice planner, Stanley controller, and pure
pursuit controller algorithms.

Unlike Stanley, Junior, Autoware and Apollo, we don't have a detailed map of
the driving course. As a result, our best option is to follow the lanes unless
a traffic sign or some other condition mandates otherwise. For the same reason,
we have to create our reference paths for our local trajectory generation
either from the online detected lane centers or according to the traffic
regulations. Meyer et al.\ \cite{Meyer2018DeepSL} study semantic lane
segmentation for mapless driving, which bears similaries to our lane detection
approach. Authors' motivation is the fact that as the high definition maps
quickly get out of date due to constructions an autonomous car should also be
able to perform basic navigation tasks without a precise map, but possibly with
a coarse map, specifically for intersections. They start with creating their
own dataset by extending Cityscape \cite{Cordts2016TheCD}. Their approach is to
annotate the road surface as ego lane, parallel lane, and opposite lane and
learn these regions with a semantic segmentation model.

Stanley controller algorithm is weak to discontinuities along the trajectory as
it directly drives towards the closest point on the trajectory. Stanley and
Junior guarantees a smooth trajectory by smoothing already known center
reference lines or post processing the output of hybrid A*. We cannot guarantee
a smooth path as we use discontinuous predefined path in response to traffic
signs and signals or due to instantaneous segmentation errors in the lane
detection. Conversely, as pure pursuit controller drives along an arc it is
less likely to be affected by discontinuities.

\chapter{Architectural Overview}
\label{chp:b3}

In this chapter, we introduce our proposed architecture of the mini driverless
car with its hardware and software components. We also present our simulation
environment, which was implemented to speed up the development and
verification processes.

\section{Hardware Configuration}

Autonomous cars are equipped with a powerful central computer, actuators and
many sensors such as IMU, Camera, and LIDAR. Our mini driverless car also has
the equivalent hardware components as shown in Figure
\ref{figure:hardware-configuration}. Details of these components are given in
Table \ref{table:hardware-configuration}.

\begin{figure}[h]
  \centering
  \includegraphics[width=.8\textwidth]{figures/hardware-configuration.pdf}
  \caption[Hardware components of the mini autonomous car]{Hardware components
    ofthe mini autonomous car.}
  \label{figure:hardware-configuration}
\end{figure}

\begin{table}[h]
  \begin{center}
    \caption[Hardware configuration]{Mini autonomous car hardware configuration
      details.}
    \label{table:hardware-configuration}
    \begin{tabular}{|c|c|}
      \hline
      \textbf{Component} & \textbf{Description} \\
      \hline
      Vehicle            & TRAXXAS SLASH 4X4 PLATINUM EDITION \\ 
      \hline
      Central Computer   & NVIDIA Jetson TX2 Developer Kit \\
      \hline
      Stereo Camera      & Stereolabs ZED Camera \\
      \hline
      2D LIDAR           & Scanse Sweep LIDAR \\
      \hline
      ESC                & Vedder Electronic Speed Controller \\
      \hline
      IMU                & SparkFun 9 DoF Razor IMU M0 \\
      \hline
      USB3.0 Hub         & USB 3.0 7-Port Hub with 2 Charging Ports UH720 \\
      \hline
      Joystick           & Logitech F710 Wireless Gamepad (940-000142) \\
      \hline
      LiPo Battery       & 4200 mAh 7,4V 25C \\
      \hline
      NiMH Battery       & MARC Power Lite 3400mAh 16V and 12V outputs \\
      \hline
    \end{tabular}
  \end{center}
\end{table}

The central computer runs various sophisticated algorithms to fuse raw data
from sensors to achieve environmental perception, select the best possible
action accordingly, and finally send speed and steering angle commands to the
ESC in order to execute the action. The ESC generates necessary electronic
signals from the commands and feeds them to servo and DC motors to control
steering angle and speed, respectively. We replace the stock ESC that ships
with the vehicle with a different speed controller known as VESC, an open
source ESC, since it is highly configurable and thus supports full control at
lower speeds \cite{Vesc2015ESC}. The hardware setup also provides a joystick
control in order to acquire dataset from a human driver and take over the
control during the autonomous drive in case of an emergency. We use two
different batteries to power the vehicle. While NiMH battery powers the
central computer and sensors through the USB 3.0 Hub, LiPo battery supplies
current to the actuators as they need a more consistent power source.

\section{Software Architecture}

We use Robot Operating System (ROS) \cite{Quigley2009ROSAO} to implement our
architecture. ROS defines notion of nodes and allows nodes to communicate
through publish/subscribe and reply/request mechanisms. Similar to previous
studies \cite{ Thrun2006StanleyTR, Montemerlo2009JuniorTS, Kato2018AutowareOB},
we also rely on publish/subscribe mechanism for the data flow between our
nodes. Nodes subscribe to the message streams from other nodes. The result of
the computation of a node is then published to the other nodes. This
asyncrononous messaging alows the software to act as a data processing
pipeline. The architecture is roughly grouped into four modules.

\begin{itemize}
  \item \textbf{sensor interface --} The sensor interface reads raw data from
    individual sensors, converts them to meaningful engineering data, then
    feeds them to the other modules.
  \item \textbf{scene interpretation --} The scene interpretation module
    makes sense of the the environment by finding traffic lanes,
    detecting obstacles, and classifying the traffic signs. Once the
    objects of interests are detected and classified, the module also
    locates these objects in the real world coordinates with respect to
    car's body frame. Then it decides on a high-level behavior that best
    fits to its current perception of the environment such as keeping a
    lane or stopping on a red light.
  \item \textbf{navigation --} The navigation module first takes the
    kinematic constraints, traffic rules, and nearby obstacles into account
    and generates an optimal trajectory to realize the high-level behavior.
    Then it computes steering angle and speed values in order to follow the
    optimal trajectory as close as possible.
  \item \textbf{actuator --} A pre-configured firmware in the VESC converts
    steering and speed commands into electronic signals to drive the
    motors.
\end{itemize}

Figure \ref{figure:software-architecture} illustrates the overall data flow in
the software modules. Out of these modules, nodes in sensor interface and
actuator modules are already provided as ROS packages. They are not implemented
but configured within the scope of this thesis.

\begin{figure}[h]
  \centering
  \includegraphics[width=.9\textwidth]{figures/software-architecture.pdf}
  \caption[Software architecture overview]{Software architecture overview.}
  \label{figure:software-architecture}
\end{figure}

Zed camera node is configured to publish visual odometry, RGB, and depth topics
at a rate of 15 Hz. Message types that flow through these topics are
well-defined in ROS. An odometry message contains position, orientation, and
linear and angular velocities with respect to the starting point. RGB image
message is a rectified color image from the zed camera, which is used for
image segmentation and classification. Because zed camera features stereo
images, it can also publish a depth image, which is used to locate the traffic
signs with respect to the body frame.

LIDAR node is configured to provide scan data to the costmap nodes in the scene
interpretation module at 5 Hz rotational speed. The costmap nodes then
publish local occupancy grid maps that indicate nearby objects. Scene
interpretation maintains two local maps in different sizes. The small map is
used in navigation module for collision avoidance. The larger map is
internally used by the scene interpretation module for behavior planning to
guide the trajectory planner before the navigation module observes the
obstacles.

Behavior planner component decides if the car should stop or in which speed
range it should move, whether it should track the lanes or follow a predefined
path. At the end, the behavior planner captures the current desired
behavior in an interpretation message and publishes to the trajectory planner.

Trajectory planner subscribes to the odometry, 3x3 local map, and
interpretation topics so as to find a collision-free, kinematically feasible,
smooth, and optimal trajectory. The trajectory message contains waypoint
locations in the odometry frame and a recommended speed for each waypoint.

Finally, given the odometry and optimal trajectory, the trajectory execution
node computes steering angles to closely follow the optimal trajectory and
publishes the recommended speed and steering commands to the actuator module.

\section{Simulation Environment}

It is not always practical to try new ideas on the target platform for several
reasons. First, it is too risky to run an updated version of the software in
the target platform as it might crash into an obstacle. Second, the batteries
have certain life time and we do not want to drain them for each immature
update to the code base. Third, deploying and testing the software on the car
is time consuming. Last but not least, running the software on the car requires
a large enough space with various traffic signs, lanes, a bridge and many other
urban conditions, which we could barely afford a few, therefore, we had no
better option than creating a simulated environment.

We used Gazebo to model our simulation world and robot car. Figure
\ref{figure:simulation-environment} shows the simulated urban area. It
simulates every case in OpenZeka MARC 2019 except that we have to manually
toggle red and green lights, and manually walk the pedestrian out of the scene.

\begin{figure}[h]
  \centering
  \begin{subfigure}[b]{0.4\linewidth}
    \includegraphics[width=\linewidth]{figures/simulation-environment1.png}
    \caption{}
  \end{subfigure}
  \begin{subfigure}[b]{0.4\linewidth}
    \includegraphics[width=\linewidth]{figures/simulation-environment2.png}
    \caption{}
  \end{subfigure}
  \caption[Simulation environment]{(a) Gazebo simulation environment top view.
    (b) The car, traffic signs, and the bridge in the simulation environment.}
  \label{figure:simulation-environment}
\end{figure}

Simulated sensors were also carefully tuned to reflect actual sensor behaviors,
but still actual camera images look blurrier. Moreover, we did not simulate the
changing lighting conditions and vibrations from the actuators. Figure
\ref{figure:camera-view} gives the actual and simulated camera views for
comparison.

\begin{figure}[h]
  \centering
  \begin{subfigure}[b]{0.4\linewidth}
    \includegraphics[width=\linewidth]{figures/actual-camera-view.jpg}
    \caption{}
  \end{subfigure}
  \begin{subfigure}[b]{0.4\linewidth}
    \includegraphics[width=\linewidth]{figures/simulated-camera-view.jpg}
    \caption{}
  \end{subfigure}
  \caption[Comparison of actual and simulated camera views]{(a) Actual camera
    view. (b) Simulated camera view.}
  \label{figure:camera-view}
\end{figure}

Despite all the peculiarities of the simulation environment, it made it
possible to quickly collect datasets without needing any additional hardware,
not even a joystick as it supports keyboard commands. We trained our
segmentation and classification models on those datasets and tested in the
simulation environment. We also developed our trajectory planning and control
algorithms in the simulation, which drastically reduced the risk of damaging
any equipment. In a nutshell, the simulation was not a replacement for the real
world, but rather served as a flexible testbed.

\chapter{Scene Interpretation}
\label{chp:b4}

\section{Environmental Perception}
\subsection{Lane Detection}
\subsection{Sign Detection}
\subsection{Obstacle Detection}

\section{Behavior Planning}
Hierarchical Finite State Machine

\chapter{Navigation}
\label{chp:b5}

\section{Trajectory Planning}
\section{Trajectory Execution}

\chapter{Experiments and Results}
\label{chp:b6}

We conducted various experiments for different scenarios in the simulation and
real miniature courses. In this chapter, we introduce our courses and datasets
collected from the courses. We evaluate our semantic segmentation and
classification deep learning models based on these datasets. We then demostrate
the behavior of our car on different traffic scenarios.

\section{Race Courses and Datasets}

Our experiments are based on four different courses. The first course is the
course provided by OpenZeka  two days before the competition, so it was not
available for use during the development. The second course is the one we
constructed in our laboratory, which is spatially no larger than the bridge
area of the actual competition course. We used it to collect simple images and
test our hardware setup to ensure the car is moving. The third one was
developed in Gazebo simulation environment similar to the competition course.
We developed the fourth course in the simulation to semi-automatically collect
extra traffic sign images. Figure \ref{figure:annotated-courses} illustrates
the courses with semantic segmentation annotations.

\begin{figure}[h]
  \centering
  \begin{subfigure}[b]{0.4\linewidth}
    \includegraphics[width=\linewidth]{figures/course1.jpg}
    \caption{Course 1}
  \end{subfigure}
  \begin{subfigure}[b]{0.4\linewidth}
    \includegraphics[width=\linewidth]{figures/course2.jpg}
    \caption{Course 2}
  \end{subfigure}
  \begin{subfigure}[b]{0.4\linewidth}
    \includegraphics[width=\linewidth]{figures/course3.jpg}
    \caption{Course 3}
  \end{subfigure}
  \begin{subfigure}[b]{0.4\linewidth}
    \includegraphics[width=\linewidth]{figures/course4.jpg}
    \caption{Course 4}
  \end{subfigure}
  \caption[Real and simulated course scenes]{Example scenes from the courses.
    Course 1 is the official competition course. Course 2 is a small track
    constructed in the laboratory. Course 3 is a simulation of Course 1. Course
    4 is used to semi-automatically collect extra traffic sign images. It is
    not used for training or testing the semantic segmentation model.}
  \label{figure:annotated-courses}
\end{figure}

We split semantic segmentation dataset into training and testing sets as shown
in Table \ref{table:semantic-segmentation-dataset}. For classification dataset,
we started with a collection of relevant signs from existing datasets
\cite{Timofte2009MultiviewTS, Stallkamp2012ManVC, Shakhuro2016RussianTS,
Serna2018ClassificationOT, MaldonadoBascn2007RoadSignDA}. Then, we expand the
dataset by automatically cropping the sign patches from the images collected
from all our four courses as the car drives itself. The new sign patches are
manually arranged and merged into the existing classification dataset. Details
of the final classification dataset is presented in Table
\ref{table:classification-dataset}.

\begin{table}[h]
  \begin{center}
    \caption[Traffic scene semantic segmentation dataset]{Traffic scene
      semantic segmentation dataset.}
    \label{table:semantic-segmentation-dataset}
    \begin{tabular}{|c|c|c|}
      \hline
      \textbf{Course}   & \textbf{Training} & \textbf{Testing} \\
      Course 1          & 2249              & 212              \\
      Course 2          & 169               & 10               \\
      Course 3          & 230               & 29               \\
      \hline
      \textbf{Total}    & 2648              & 251              \\
      \hline
    \end{tabular}
  \end{center}
\end{table}


\begin{table}[h]
  \begin{center}
    \caption[Traffic sign classification dataset]{Traffic sign classification
      dataset.}
    \label{table:classification-dataset}
    \begin{tabular}{|c|c|c|c|}
      \hline
      \textbf{Class}     & \textbf{Training} & \textbf{Testing} & \textbf{Auto-cropped} \\
      \hline
      PedestrianCrossing & 704               & 89               & 470 \\
      KeepLeft           & 844               & 83               & 594 \\
      LooseGravel        & 533               & 60               & 593\\
      NoEntry            & 502               & 83               & 306 \\
      Parking            & 497               & 90               & 237 \\
      ParkingSlot        & 1730              & 60               & 1790 \\
      RoadWork           & 1741              & 86               & 237 \\
      StraightOrRight    & 1106              & 89               & 848 \\
      TrafficLightGreen  & 422               & 76               & 192 \\
      TrafficLightRed    & 1147              & 72               & 1095 \\
      TurnLeft           & 817               & 93               & 502 \\
      Negative           & 984               & 60               & 1044 \\
      \hline
      \textbf{Total}     & 11027             & 961              & 7908 \\
      \hline
    \end{tabular}
  \end{center}
\end{table}

\section{Perception Evaluation}

We evaluate our semantic segmentation model using IoU, Precision, Recall, and
F1 scores for each class given in Table
\ref{table:semantic-segmentation-test-results}. Despite the fact that we used
our own dataset, we compare our lane segmentation results to
\cite{Barnes2016FindYO} and  \cite{Meyer2018DeepSL} as they are mostly
relevant. For ego lane, which is the main enabler for driving,
\cite{Barnes2016FindYO} achieves up to 85\% IoU and \cite{Meyer2018DeepSL}
achieves 80\% IoU. We achieved 88\% IoU on our dataset. However, note that our
lanes are more obvious compared to real traffic scenes. In real scenes, lane
lines are often obscured or worn-out. For the same reason, our metrics for
neighboring lanes (i.e., right and left lanes) are better the corresponding
metrics given by \cite{Meyer2018DeepSL} for opposite and parallel lanes. Our
ego lane precision, recall and F1 scores are also comparable to the results
provided by these studies.

One can notice from Table \ref{table:semantic-segmentation-test-results} that
the right lane IoU is considerably smaller than that of the other lanes and
road side. This is because right lane is underrepresented in the dataset as we
most of the time drive the car on the right lane, effectively using it as the
ego lane. The same reason is also applicable for traffic signs. Traffic signs
are small and rare often located towards the edge of the images.

\begin{table}[h]
  \begin{center}
    \caption[Semantic segmentation test results]{Semantic segmentation
      test results.}
    \label{table:semantic-segmentation-test-results}
    \begin{tabular}{|c|c|c|c|c|}
      \hline
      \textbf{Class} & \textbf{IoU} & \textbf{Precision} & \textbf{Recall} & \textbf{F1}  \\
      Background     & 94.43\%      & 97.81\%            & 96.43\%         & 97.12\%      \\
      Ego Lane       & 88.48\%      & 92.29\%            & 93.18\%         & 92.73\%      \\
      Left Lane      & 83.52\%      & 87.83\%            & 94.49\%         & 91.04\%      \\
      Road Side      & 71.97\%      & 78.49\%            & 89.58\%         & 83.67\%      \\
      Right Lane     & 51.32\%      & 55.60\%            & 90.19\%         & 68.79\%      \\
      Traffic Sign   & 58.12\%      & 79.52\%            & 71.81\%         & 75.47\%      \\
      \hline
      \textbf{All}   & 74.64\%      & 81.92\%            & 89.28\%         & 84.80\%      \\
      \hline
    \end{tabular}
  \end{center}
\end{table}

For traffic sign detection, we run a classification model on the top of the
semantic segmentation that proposes regions to the classification
network. Table \ref{table:classification-test-results} presents the detailed
performance metrics of our 97.45\% accuracy classifier.

We find adding an extra negative class helpful to improve overall
classification performance. Nonetheless, \textit{Negative} class also has false
negatives lowering its recall as given in Table
\ref{table:classification-test-results}. In other words, a non-traffic sign
region can still be classified as a traffic sign as shown in Figure
\ref{figure:real-course-detection-bad}.

\begin{table}[h]
  \begin{center}
    \caption[Traffic sign classification test results]{Traffic sign
      classification test results.}
    \label{table:classification-test-results}
    \begin{tabular}{|c|c|c|c|}
      \hline
      \textbf{Class}     & \textbf{Precision} & \textbf{Recall}  & \textbf{F1-score} \\
      \hline
      KeepLeft           & 100.00\%           & 100.00\%         & 100.00\% \\
      LooseGravel        & 100.00\%           & 100.00\%         & 100.00\% \\
      NoEntry            & 95.40\%            & 100.00\%         &  97.65\% \\
      Parking            & 98.86\%            & 96.67\%          &  97.75\% \\
      ParkingSlot        & 84.51\%            & 100.00\%         &  91.60\% \\
      PedestrianCrossing & 97.72\%            & 69.63\%          &  97.17\% \\
      RoadWork           & 95.50\%            & 98.84\%          &  97.14\% \\
      StraightOrRight    & 100.00\%           & 97.75\%          &  98.86\% \\
      TrafficLightGreen  & 98.68\%            & 97.37\%          &  98.01\% \\
      TrafficLightRed    & 100.00\%           & 97.22\%          &  98.60\% \\
      TurnLeft           & 98.91\%            & 97.85\%          &  98.37\% \\
      Negative           & 100.00\%           & 85.00\%          &  91.90\% \\
      \hline
    \end{tabular}
  \end{center}
\end{table}

In order to evalute our traffic sign detection performance we use mAP measure
defined in PASCAL VOC 2012 competition \cite{Everingham2010ThePV}. Detected
signs are first sorted by decreasing classification confidance and matched up
with ground truth signs. If the matched pair achieves IoU $\ge$ 0.5 and has the
same class label, the match is considered to be a true positive. Then using
this information, we build a precision/recall curve with monotonically
decreasing precision. Next, AP for the class label is computed by numeraically
integrating the area under the curve. Finally, we compute mAP as the mean of
all APs \cite{Cartucho2019MAP}. Figure \ref{figure:average-precision} shows APs
for all classes and their false positive rates excluding \textit{Negative}
class.

\begin{figure}[h]
  \centering
  \begin{subfigure}[b]{0.45\linewidth}
    \includegraphics[width=\linewidth]{figures/experiments/mAP.png}
    \caption{}
  \end{subfigure}
  \begin{subfigure}[b]{0.45\linewidth}
    \includegraphics[width=\linewidth]{figures/experiments/detection-results-info.png}
    \caption{}
  \end{subfigure}
  \caption[Evaluation of traffic sign detection and recognition]{(a)Average
    precision for traffic sign detection. (b) True and false prediction rates
    for each sign class.}
  \label{figure:average-precision}
\end{figure}


Figure \ref{figure:real-course-detection-good} and Figure
\ref{figure:real-course-detection-bad} visualize segmentation and sign
detection results on the test images taken from real courses, Course 1 and
Course 2.

\begin{figure}[h]
  \centering
  \begin{subfigure}[b]{0.45\linewidth}
    \includegraphics[width=\linewidth]{figures/experiments/real/keepleft.jpg}
  \end{subfigure}
  \begin{subfigure}[b]{0.45\linewidth}
    \includegraphics[width=\linewidth]{figures/experiments/real/loosegravel.jpg}
  \end{subfigure}
  \begin{subfigure}[b]{0.45\linewidth}
    \includegraphics[width=\linewidth]{figures/experiments/real/noentry.jpg}
  \end{subfigure}
  \begin{subfigure}[b]{0.45\linewidth}
    \includegraphics[width=\linewidth]{figures/experiments/real/parking.jpg}
  \end{subfigure}
  \begin{subfigure}[b]{0.45\linewidth}
    \includegraphics[width=\linewidth]{figures/experiments/real/parkingslot.jpg}
  \end{subfigure}
  \begin{subfigure}[b]{0.45\linewidth}
    \includegraphics[width=\linewidth]{figures/experiments/real/pedestrian-crossing.jpg}
  \end{subfigure}
  \begin{subfigure}[b]{0.45\linewidth}
    \includegraphics[width=\linewidth]{figures/experiments/real/straightorright.jpg}
  \end{subfigure}
  \begin{subfigure}[b]{0.45\linewidth}
    \includegraphics[width=\linewidth]{figures/experiments/real/trafficlightred.jpg}
  \end{subfigure}
  \caption[True sign detections on real courses]{Sample true sign
    detections on real courses.}
  \label{figure:real-course-detection-good}
\end{figure}


\begin{figure}[h]
  \centering
  \begin{subfigure}[b]{0.45\linewidth}
    \includegraphics[width=\linewidth]{figures/experiments/real/fn-pedestrian.jpg}
  \end{subfigure}
  \begin{subfigure}[b]{0.45\linewidth}
    \includegraphics[width=\linewidth]{figures/experiments/real/fp-noentry1.jpg}
  \end{subfigure}
  \begin{subfigure}[b]{0.45\linewidth}
    \includegraphics[width=\linewidth]{figures/experiments/real/fp-noentry2.jpg}
  \end{subfigure}
  \begin{subfigure}[b]{0.45\linewidth}
    \includegraphics[width=\linewidth]{figures/experiments/real/fp-noentry3.jpg}
  \end{subfigure}
  \begin{subfigure}[b]{0.45\linewidth}
    \includegraphics[width=\linewidth]{figures/experiments/real/fp-roadwork.jpg}
  \end{subfigure}
  \begin{subfigure}[b]{0.45\linewidth}
    \includegraphics[width=\linewidth]{figures/experiments/real/fp-trafficlight.jpg}
  \end{subfigure}
  \begin{subfigure}[b]{0.45\linewidth}
    \includegraphics[width=\linewidth]{figures/experiments/real/fp-turnleft1.jpg}
  \end{subfigure}
  \begin{subfigure}[b]{0.45\linewidth}
    \includegraphics[width=\linewidth]{figures/experiments/real/fp-turnleft2.jpg}
  \end{subfigure}
  \caption[False sign detections on real courses]{Sample false sign
    detections on real courses.}
  \label{figure:real-course-detection-bad}
\end{figure}


Figure \ref{figure:driving-comparison} presents plots to compare autonomous
driving with a manual drive on Course 3. When the car sees a red light it
decelerates and stops at time $t = 20s$. Sudden speed jump during the
deceleration is due to an instantaneous lose of the red light. When it gets
closer to the red light, it detects it back and finally stops. At $t = 40s$ the
car detects road work and transitions to its slow speed state and performs a
left lane change. At $t = 60s$, it reaches back to the right lane and starts to
take a left turn. At $t = 70s$, it once more transtions to slow speed due to a
loose gravel sign detection. Between $85s$ and $95s$, the car overtakes a
static obstacle blocking the left lane and it is manually stopped when it is
back to the right lane.

\begin{figure}[h]
  \centering
  \begin{subfigure}[b]{0.45\linewidth}
    \includegraphics[width=\linewidth]{figures/experiments/speed-plot.png}
    \caption{Speed vs. time plot.}
  \end{subfigure}
  \begin{subfigure}[b]{0.45\linewidth}
    \includegraphics[width=\linewidth]{figures/experiments/steering-plot.png}
    \caption{Steering vs. time plot.}
  \end{subfigure}
  \begin{subfigure}[b]{0.45\linewidth}
    \includegraphics[width=\linewidth]{figures/experiments/position-plot.png}
    \caption{Position x-y plot.}
  \end{subfigure}
  \caption[Comparison between autonomous driving and a human driver]{Comparison
    between autonomous driving and a human driver.}
  \label{figure:driving-comparison}
\end{figure}


Figures \ref{figure:normal-driving}, \ref{figure:lane-change}, and
\ref{figure:stop} further illustrate various driving scenarios by proving 3D
point cloud of traffic scenes and processed front view camera images.


\section{Driving Tasks}

\begin{figure}[h]
  \centering
  \begin{subfigure}[b]{0.45\linewidth}
    \includegraphics[width=\linewidth]{figures/experiments/lane-following-img.png}
  \end{subfigure}
  \begin{subfigure}[b]{0.45\linewidth}
    \includegraphics[width=\linewidth]{figures/experiments/lane-following-pc.png}
  \end{subfigure}
  \begin{subfigure}[b]{0.45\linewidth}
    \includegraphics[width=\linewidth]{figures/experiments/straight-or-right-img.png}
  \end{subfigure}
  \begin{subfigure}[b]{0.45\linewidth}
    \includegraphics[width=\linewidth]{figures/experiments/straight-or-right-pc.png}
  \end{subfigure}
  \begin{subfigure}[b]{0.45\linewidth}
    \includegraphics[width=\linewidth]{figures/experiments/loose-gravel-img.png}
  \end{subfigure}
  \begin{subfigure}[b]{0.45\linewidth}
    \includegraphics[width=\linewidth]{figures/experiments/loose-gravel-pc.png}
  \end{subfigure}
  \caption[Straight road driving scenarios]{Straight road driving scenarios.
    For the first row, the car is configured to follow lanes and detect signs
    without taking any action for detections. Note that the perception
    component is not trained to drive on this course, but it still manages to
    drive.}
  \label{figure:normal-driving}
\end{figure}

\begin{figure}[h]
  \centering
  \begin{subfigure}[b]{0.45\linewidth}
    \includegraphics[width=\linewidth]{figures/experiments/turn-left-img.png}
  \end{subfigure}
  \begin{subfigure}[b]{0.45\linewidth}
    \includegraphics[width=\linewidth]{figures/experiments/turn-left-pc.png}
  \end{subfigure}
  \begin{subfigure}[b]{0.45\linewidth}
    \includegraphics[width=\linewidth]{figures/experiments/parking-img.png}
  \end{subfigure}
  \begin{subfigure}[b]{0.45\linewidth}
    \includegraphics[width=\linewidth]{figures/experiments/parking-pc.png}
  \end{subfigure}
  \caption[Sharp turning scenarios]{Sharp turning scenarios.}
  \label{figure:sharp-turns}
\end{figure}

\begin{figure}[h]
  \centering
  \begin{subfigure}[b]{0.45\linewidth}
    \includegraphics[width=\linewidth]{figures/experiments/construction-zone-img.png}
  \end{subfigure}
  \begin{subfigure}[b]{0.45\linewidth}
    \includegraphics[width=\linewidth]{figures/experiments/construction-zone-pc.png}
  \end{subfigure}
  \begin{subfigure}[b]{0.45\linewidth}
    \includegraphics[width=\linewidth]{figures/experiments/overtaking1-pc.png}
  \end{subfigure}
  \begin{subfigure}[b]{0.45\linewidth}
    \includegraphics[width=\linewidth]{figures/experiments/overtaking2-pc.png}
  \end{subfigure}
  \begin{subfigure}[b]{0.45\linewidth}
    \includegraphics[width=\linewidth]{figures/experiments/overtaking3-pc.png}
  \end{subfigure}
  \begin{subfigure}[b]{0.45\linewidth}
    \includegraphics[width=\linewidth]{figures/experiments/overtaking4-pc.png}
  \end{subfigure}
  \caption[Lane change scenarios]{Lane change scenarios.}
  \label{figure:lane-change}
\end{figure}

\begin{figure}[h]
  \centering
  \begin{subfigure}[b]{0.45\linewidth}
    \includegraphics[width=\linewidth]{figures/experiments/pedestrian-crossing-stop-img.png}
  \end{subfigure}
  \begin{subfigure}[b]{0.45\linewidth}
    \includegraphics[width=\linewidth]{figures/experiments/pedestrian-crossing-stop-pc.png}
  \end{subfigure}
  \begin{subfigure}[b]{0.45\linewidth}
    \includegraphics[width=\linewidth]{figures/experiments/pedestrian-crossing-go-img.png}
  \end{subfigure}
  \begin{subfigure}[b]{0.45\linewidth}
    \includegraphics[width=\linewidth]{figures/experiments/pedestrian-crossing-go-pc.png}
  \end{subfigure}
  \begin{subfigure}[b]{0.45\linewidth}
    \includegraphics[width=\linewidth]{figures/experiments/red-light-stop-img.png}
  \end{subfigure}
  \begin{subfigure}[b]{0.45\linewidth}
    \includegraphics[width=\linewidth]{figures/experiments/red-light-stop-pc.png}
  \end{subfigure}
  \begin{subfigure}[b]{0.45\linewidth}
    \includegraphics[width=\linewidth]{figures/experiments/green-light-go-img.png}
  \end{subfigure}
  \begin{subfigure}[b]{0.45\linewidth}
    \includegraphics[width=\linewidth]{figures/experiments/green-light-go-pc.png}
  \end{subfigure}
  \begin{subfigure}[b]{0.45\linewidth}
    \includegraphics[width=\linewidth]{figures/experiments/parking-slot-img.png}
  \end{subfigure}
  \begin{subfigure}[b]{0.45\linewidth}
    \includegraphics[width=\linewidth]{figures/experiments/parking-slot-pc.png}
  \end{subfigure}
  \caption[Stopping scenarios]{Stopping scenarios.}
  \label{figure:stop}
\end{figure}

\chapter{Conclusion}
\label{chp:b7}

This thesis aimed to investigate autonomous car software architectures for
urban driving scenarios. Based on the existing studies, decomposed
architectures prove to be more successful in city traffic compared to
end-to-end solutions. As a budget-friendly alternative, we developed a mini
autonomous car with a decomposed architecture for urban scenarios including
seven different traffic signs, traffic signals, bridge, overtaking a stationary
car, and parking with no predefined map. These scenarios together makes the
problem too complicated to be addressed with the existing mini autonomous car
solutions.

Due to lack of traffic scene segmentation datasets for mini cars, we started
with creating a dataset. Then we trained a U-net based model to learn ego lane,
right lane, left lane, road side, and traffic signs semantics from the camera
images. Learning lane semantics for right and left lanes along the with ego
lane enabled the trajectory planner to implement a lane change policy such that
it assigns more cost to occupying left lane when the right lane is available.
We showed that we achieved 88\% IoU for the ego lane, which is comparable to
the related work.

For the traffic sign and signal classification, we first implemented a region
proposal algorithm relying on precision of the traffic sign segmentation. Then
we merged existing relevant datasets for our scenarios and augmented it by
cropping the sign region proposals. Finally, we trained another relatively
small deep learning model with the final dataset and achieved 97\% accuracy.

We implemented a finite state machine for invoking different behaviors of our
car. This behavior planner interprets environmental perception for the
trajectory planner. The trajectory planner finds an optimum trajectory for
desired behavior such as target speed or target point to stop, for example, due
to a red light.

Our work can be improved in a number of ways. For the perception side, the sign
detection clearly has room for improvement. There are many different approaches
to sign detection with varying resource demands. These approaches can be
evaluated on the target platform, by modifying the approaches if necessary. An
interesting approach would be to train a multi-task model for learn lane
semantics as well as detecting and classifying traffic signs. For the planning
side, although our trajectory planner works well for structural environments,
it is not suitable for unstructured environments such as parking lots.
Algorithms such as hybrid A* and state lattice planner would be better choices
for parking lot navigation. In addition, we only deal with static obstacles in
the current study. For responding to dynamic obstacles, we need an additional
prediction component that estimates the trajectories of moving objects. The
trajectory planner should also take these trajectories into account. Moreover,
the trajectory planner should be extended with distance keeping mode so that
the behavior planner can choose to follow moving vehicles with a safety margin.
Last, we only follow lanes and directions provided by traffic signs with no
sense of global direction. Integration of a coarse map would enable the car
to make global plans between current and destination locations so that it could
evaluate multiple directions to find the shortest path to the destination.



\input{references.tex}

%
% References in Bibtex format goes into below indicated file with .bib extension
%\bibliography{thesis_references}
% You can use full name of authors, however most likely some of the Bibtex entries you will find, will use abbreviated first names
% If you don't want to correct each of them by hand, you can use abbreviated style for all of the references

% \bibliographystyle{abbrv}

% if you have more that one appendix, then use \appendices, otherwise use
% \appendix
% \input{appendix1/appendix1.tex}
\end{document}
