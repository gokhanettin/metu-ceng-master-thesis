\chapter{Conclusion}
\label{chp:b7}

This thesis aimed to investigate autonomous car software architectures for
urban driving scenarios. Based on the existing studies, decomposed
architectures prove to be more successful in city traffic compared to
end-to-end solutions. As a budget-friendly alternative, we developed a mini
autonomous car with a decomposed architecture for urban scenarios including
seven different traffic signs, traffic signals, bridge, overtaking a stationary
car, and parking with no predefined map. These scenarios together makes the
problem too complicated to be addressed with the existing mini autonomous car
solutions.

Due to lack of traffic scene segmentation datasets for mini cars, we started
with creating a dataset. Then we trained a U-net based model to learn ego lane,
right lane, left lane, road side, and traffic signs semantics from the camera
images. Learning lane semantics for right and left lanes along the with ego
lane enabled the trajectory planner to implement a lane change policy such that
it assigns more cost to occupying left lane when the right lane is available.
We showed that we achieved 88\% IoU for the ego lane, which is comparable to
the related work.

For the traffic sign and signal classification, we first implemented a region
proposal algorithm relying on precision of the traffic sign segmentation. Then
we merged existing relevant datasets for our scenarios and augmented it by
cropping the sign region proposals. Finally, we trained another relatively
small deep learning model with the final dataset and achieved 97\% accuracy.

We implemented a finite state machine for invoking different behaviors of our
car. This behavior planner interprets environmental perception for the
trajectory planner. The trajectory planner finds an optimum trajectory for
desired behavior such as target speed or target point to stop, for example, due
to a red light.

Our work can be improved in a number of ways. For the perception side, the sign
detection clearly has room for improvement. There are many different approaches
to sign detection with varying resource demands. These approaches can be
evaluated on the target platform, by modifying the approaches if necessary. An
interesting approach would be to train a multi-task model for learn lane
semantics as well as detecting and classifying traffic signs. For the planning
side, although our trajectory planner works well for structural environments,
it is not suitable for unstructured environments such as parking lots.
Algorithms such as hybrid A* and state lattice planner would be better choices
for parking lot navigation. In addition, we only deal with static obstacles in
the current study. For responding to dynamic obstacles, we need an additional
prediction component that estimates the trajectories of moving objects. The
trajectory planner should also take these trajectories into account. Moreover,
the trajectory planner should be extended with distance keeping mode so that
the behavior planner can choose to follow moving vehicles with a safety margin.
Last, we only follow lanes and directions provided by traffic signs with no
sense of global direction. Integration of a coarse map would enable the car
to make global plans between current and destination locations so that it could
evaluate multiple directions to find the shortest path to the destination.
